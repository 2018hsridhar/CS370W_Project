\documentclass{article}
\begin{document}

%%%%%%%% PRELIMS %%%%%%%%%%
%%%%%%%%%% paper story / abstract %%%%%%%%%%%%%%%%%%%%
\section{The paper story}

%%%%%%%%%%% the five c's %%%%%%%%%%%%%%%%
\section{The five C's}
\begin{itemize}
	\item Clarity - Well written.
	\item Category - Description of a method that tackles 3D model creation, based on alignment of 2 nonoverlapping partial scans
	\item Contributions - new method for aligning and stitching 2 nonoverlapping meshes, based just on boundary information. Breakable into 3 main contributions.
	\item Correctness - Assumptions appear to be valid
	\item Context/Related to :: Alignment scheme algorithms ( RigidIcp), our research project
\end{itemize}

%%%%%%%%%%% key/unknown vocab %%%%%%%%%%
\section{Key or Unknown Vocabulary}
\begin{enumerate}[(i)]
	\item Digitation of objects -
	\item Range Sensor
	\item Point cloud filtering approaches [ Kazhdan ] - 
	\item Convex Hull Approx - 
	\item Ball Pivoting - is this what I'm doing? 
	\item pos procesing
\end{enumerate}

%%%%% THE DEPTH REPORT %%%%%%%%%%
%%%%%%%%%%% introduction %%%%%%%%%%%%%%%%
\section{Summary of Introduction}
	% JANSOM is a method that tackles 3D model creation, based on alignment of 2 nonoverlapping partial scans.
	%  Easy to obtain non-overlapping meshes
	% Uses only boundary geometry for alignment
	% Can be used for hole-filing
	% ASSUMED that boundaries of scans are the same geometric struct, from different viewpoint. Formulates alignment as minization of total edge len. Recognizes the difficulties associated with this assumption ( lack of a priori knowledge, lack of good boundaries)
	% Contributions to address limitations are :
	% 	- introducion of cost function to penalize edge lens and intersections
	% 	- solving assisgments, via constraint generation (change of vars), changes from a combinatorial to disc Linear Programming problem
	% 	- a new stitching algo

\section{Summary of Related Works, Section 2}
% Little work has been done in the case of zero overlap
% Use of range scans for 3D obj modelling - motivation for mesh stitching 
% 	- [ Turk and Levoy, 1994 ] - 3 step alignment + stitch algo - uses ICP to alignA - fails here, due to lack of overlap ( no ICP )
% 	- [ Paulay et Al, 2005 ] - use parts of other objs for hole filing - nonrigid alignment - then use Turk's approach
% 	- [ Borodn et Al, 2002 ] - assumes previously registered meshes. stitches by minimzing lengths of new edges, BUT does create and delete boundary vertices
% Overall, these approaches ... kinda assume a good initial alignment, or overlap. hence, they fail to solve the problem.

\section{Mesh Alignment}
% Add new, non-self interesting, edges, interpolating the boundaries, s.t. total length of all edges is minimized. 
%  Solves cost function J = J_1 + J_2 ( penalize ( len + intersection ))
% 


\section{References of Interest}
	%#TODO

%%% JANSOM - AN OVERVIEW REPROT %%%%%%%%%%%%
\section{Assumptions}

\section{Limitations}
\begin{enumerate}[(i)]
	\item wer
\end{enumerate}

\section{What stands out}
\begin{itemize}
	\item wer
\end{itemize}

\section{Constraints enforced by resesarch}
\begin{itemize}
\end{itemize}

\section{Questions on JASNOM paper}
\begin{itemize}

\end{itemize}

\section{Concerns about JASNOM paper}
\begin{itemize}

\end{itemize}

\section{Thoughts on MetaLearning}
\begin{itemize}
	\item How do I tell if a paper's methodologies and results are reproducible or not? Sure, the math is easy to assess, but the exp setup? 
	\item How do I tell if enough , or too much , detail has been imparted into specific sections?
	\item How to notice open issues that are not ever handled again (Krishnamurti)?
\end{itemize}

\section{Things learned in today's reading + Keenan Crane's Heat Flow Presentation}
\begin{itemize}
	\item Shriram Krishnamuti's paper:: It is also difficult from the sparse description to determine exactly why the outcome was as it was. While the authors deserve praise for laying out all the events in a total order, we are not given enough detail about what happens at each step to be able to reproduce the outcome \\
		- suggest to me that I should state while Keenan Crane's paper was well laid-out, his work isn't exactly the most reproducible
	\item TODO
\end{itemize}

\end{document}

% nice note :: due to planar topology, range images INDUCE an intrinsic mesh in point clouds, but do not represent the whole object.


