\documentclass{article}
\begin{document}

\title*{My Grahics Thought Process}
\author{Hari Sridhar}
\date{\today}
\newpage

\part{Feb 07 2016}
\section{Software Development Practices}
\begin{itemize}
	\item having a helpers.h, helpers.cpp, for a hodgepot of useful functions around the codebase, seems rather useful. 
	\\ Later refactoring can be implemented to set some of these methods private, or belonging to various classes
	\item The need for good namespaces, to undersatnd where things logically fit, makes more sense to me now!
	\item For things that belong in a hodgepot class or header file ( i.e. globals.h, helpers.h), ALWAYS keep a running tab on where things are used, so that if there is a potential to refactor those codebases, then they can be succesfully refactored l8r down that road.
	\item For inclusion of useful, const global vars ... use the "glob.h, glob.cpp" pattern - see http://stackoverflow.com/questions/1045501/how-do-i-share-variables-between-different-c-files
	\item Always incorporate a .gitignore. It turns out that you seldom want to store your project build, or compiler files, it GitHub all the time. Be more efficient with this. Same for *.swp files , or some libs
	\item Start using branches and merges, in development ( every time u change ... no matter size ... BRANCH!). The MASTER must always be a working copy!  ( see https://www.atlassian.com/git/tutorials/using-branches ) 
	\item Man ... it'd be nice to have a UTILITIES directory, that consists of a bunch of nice helper methods and programs :-). But, this increases development a bit :-(.
\end{itemize}

\section{Refactoring Large Codebases}
\subsection{A need for a testing directory/directory tree}
This issue arose when refactoring lots of the graphics code
\\ But I noticed that I need to include some code to consistently test that components still work, as expected, in case I end up changing major components!
\\ More ever, I need to cincorporate testing in a nice way ( i.e. each piece of testing, possesss its own viewer, and can be well-modified with a couple of nice, user based parameters )
\\ TBH, it should be easy converting the original main methods to a suite of testing methods. I should just put it elsewhere though! 
\newpage

\part{Feb 08 2016}
% seems that this version of LaTex ( pdfLatex), does not support \chapter{}
\section{Notes on CGal}
\begin{itemize}
	\item Seems that LibIgl's igl/copyLeft, provide ssome nice methods that work in sync with CGAL
	\item What about any methods that interface between data types, in CGAL, versus data types, in LibIgl? 
	\item STOP ... confusing triangultion, with remeshing. Remeshing, does use triangulation, BUT they are not the same thing! 
	\item Appears that CMakeLists.txt needed to be enabled to find CGAL, but I currently do NOT posses CGAL :-(
	\item Do I also need to go about configuring CGAL?? GAAH this will be anoying as fuck !
		see http://doc.cgal.org/Manual/3.4/doc_html/installation_manual/Chapter_installation_manual.html
\end{itemize}

\section{More things to look at in depth}
\begin{itemize}
	\item What is Remeshing of faces, according to intersection detection and construction?
	\item What is a core library ( CGAL vs CGAL_Core, Eigen vs Eigne_Core)?
	\item Do be able to set up an install CGAL utilities thyself ( for now, I will gladly skip this step ). Same for a standalone ap that just needs Eigen ( ... I did this once, naa ) ?? 
	\item seriously ... if u refactor, make sure it #()@*U works
	\item Keep trying to learn more about CMake, and make files
	\item why is it that ignoring the inclusion of specific header files ( i.e. #include <iostream> ), can end up affecting a lot of other code on the way down? This seems SO stupid!
	\item Note :: always set up header files, and method signatures ... you NEVER want to refactor these l8r down the road!
	\item Note :: do I really need a "using namespace std" put in every single file! This is rather redundant.
	\item Note :: always be sure that you know the header file of where a method is incoming from ... else, you might just happen to have othre header files GLADLY "hide" this away from you! 
\end{itemize}

\section{Thoughts}
\begin{itemize}
	\item Always ... write up a report, of progress that was made, roadblocks, things learnt, etc., 
	\item To incorporate l8r for paper writing, write up a short segment on the remeshing approach, as you IMPLEMENT remeshing
	\item Mark down a timeline of goals, that you want to accomplish! Seriously!
	\item What exactly is mesh subdivision ... not fully familiar yet, atm!
	\item Use the term \textit{Geometric algorithms}, to describe one's knowledge of graphics algos
	\item always a good idea to take out unsused header files, and using namespaces ( just declare like <namespace>::<class>)
	\item When it comes to refactoring ... refactor, s.t. you achieve a specific goal. Refactoring still comes to to Knuths "Premature optimization", and it's best to keep making progress on your program, then optimize. I guess refactoring is a good thing to do if you feel like doing nothing else, but in a sense, it's kinda liek cleaning your room, vs doing your hw.
	\item Err :: parameter reshadowing = redeclaration of a var
	\item TRY to avoid unused variables ... while not compile time errs, they do mess up code quality :-(
	\item One idea - try to avoid dumping a lot of stuff into header files ( *according to Brady ). In my view, header files seem useful ONLY for (i) global vars (ii) namespace-method fwd declarations
	\item \textbf{TBD} 
\end{itemize}

\section{Github madness - dealing with branches}
\subsection{A concise overview}
\begin{itemize}
	\item REMEMBER : brnaches = pointers, NOT copies, to commits!
	\item The master / main brnach should always be stable code.
	\item https://www.atlassian.com/git/tutorials/using-branches - for setting up git branches
	\item http://stackoverflow.com/questions/14628540/why-do-i-only-see-my-master-branch-on-github
	\item Seems that branches are very useful, say, IF I want to develop two independent features concurrently. Also useful to start different feature histories, before merging to main.
	\item \textit{How recursively deep should my branches be?} 
	\item Development MUST always take plane on a branch, NEVER on a detahed HEAD. You need to have a REF to new commits! 
	\item You can go to nonHEAD nodes, to old commits, in a read only mode


	\item http://stackoverflow.com/questions/8965781/update-an-outdated-branch-against-master-in-a-git-repo 
	\item \texit{How to ensure that I can stash my own changes, when master updates, but NOT lose my changes? unsure atm!}
	\item http://stackoverflow.com/questions/20808892/git-diff-between-current-branch-and-master-but-not-including-unmerged-master-com
\end{itemize}

\subsection{New Terminology to familiarize with}
\begin{itemize}
	\item Project history
	\item Staging Area
	\item Working dir
	\item Checkout
	\item Merge
	\item Stable vs Unstable code
	\item Header-only libraryes ( versus other libs ) 
\end{itemize}


\end{document}




