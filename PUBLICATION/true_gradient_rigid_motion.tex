\documentclass{article}
\begin{document}

% I need to find an "apt" title for this algorithm
% are these forces along the "gradient" of the MIS?

% detail the benefit and drawbacks of SGD
% explain the motivation for this scheme
% introduce specific defintions to the reader
	% be specific ... how is a rigid body defined in this setting ( is it simple, connected, closed? ) 
	% what is an external force? vs configurational force? does the reader need to know about this?

%%%%%%%%%%%%%%%%%%%%%%%%%%%%%%%%%
%%%%%%%%% Introduction %%%%%%%%%%
%%%%%%%%%%%%%%%%%%%%%%%%%%%%%%%%%
\section{Rigid Body Free Motion Based Alignment}

\indent Although Stoichastic Gradient Descent ( SGD ) generally guarantees convergence to a true solution of the posed alignment problem, it unfortunately possesses a myriad of drawbacks. Due to the large sample size require at each timestep,  for especially large inputs, SGD proves slow, and performs suboptimally. More ever, it lacks the guarantee of convergence to a global minima of the energy function.
	
\indent On the other hand, we can utilize an alignment scheme, based on a physical interpretation of stitching, that performs significantly better than SGD. Let us treat the two inputs to be aligned as \textit{three-dimensional rigid bodies}. We define such ridig bodies as solid regions $\Omega$, present in their ambient 3-dimensional Euclidean space, that can translate and rotate within the space, but cannot stretch, bend, or undergo deformations except for rigid motions. In our case, we assume that rigid bodies possess constant density $\rho$ within the body's area, and that said ares lies inside a single, simple, but possibly non-convex trimesh in the space.  

\indent Let $\{ v_{i,1},v_{i,2},\cdots,v_{i,n} \}$ denote boundary vertices of the two rigid bodies $B_1,B_2, i \in \{ 1,2 \}$. Since rigid bodies are constrained to only translations and rotations, we require only three degrees of freedoms to parameterize their motion : three for the translation, and one for the angle of rotation of the body. Let $q = (t_x,t_y,t_z,\theta)$ represent such quantities. Let us also pick a "center" point $\bar{c}$, and a \textit{undeformed} pose, consisting of an assignment of a position $\bar{v_i}$ to each vertex. Let us also set up two different coordinate systems; one correspondings to the body in its rest pose ( \textit{body coordinates} ), and one to the body in a deformed pose ( \textit{world coordinates} ).

% in a sense, i could say I'm moving along a vector field, defined specifically on the boundary edges, right?? ... that interpretatiion seems unsuitable though..
% .... TBH, you care only about the 2D case! ... SINCE your boundaries are 2 dimensinal objects, in this case!  ... nope, they are 2D boundaries, in a 3D setting. your rotation and translatinos are both 3D !

% ... WAIT, HOW do I make sure that my rotation and taranslation matrices suffice the correct property?? ... tbh, I don't have to worry about this, right? ... just apply those conversion formulaes [ axis,angle -> quat -> rot matrix ] !  

\indent We now set up a theoretical formulation describing the motion of rigid bodies. For any point $\bar{v}$ in body coordinates, the following provides a formula describing the corresponding position $v(q)$ in world coordinates, as a function of the configuration $q$. 
\begin{equation}
	v(q) = R_{\theta}(\bar{v} - \bar{c}) + \bar{c} + t
\end{equation}
where $R_{\theta}w$ represents a rotation, of the vector $w$, counterclockwise by $\theta$ radians. It should be noted that $R_{\theta}$ is linear in $w$, and has the matrix form
%%%% #TODO :: edit typesetting
%\begin{equation}
%	R_{\theta} = \begin{pmatrix}
%            \cos(\theta) & -\sin(\theta) \\
%            \sin(\theta) & \cos(\theta)
%        \end{pmatrix}
%\end{equation}
To simplify the formula, we set the rest pose of the rigid body such that $\bar{c}$ lies at the origin. Thus, we get 
\begin{equation}
	v(q) = R_{\theta}(\bar{v}) + t
\end{equation}  
Since the natural interpretation of the rigid body's center point it its \textit{center of mass}, or equivalently, \textit{centroid}, we set $\mathbf{\bar{c}}$ as : 
\begin{equation}
	\bar{c}_{cm} = \frac{1}{A(\Omega)} \int_{\Omega}(x,y)dA
\end{equation}
where $A(\Omega)$ denotes the rigid body's area, and $dA = dxdy$ denotes the surface area element. This $\bar{c}_cm$ is computed using Stokes Theorem ( refer to appendix, for details ). \\
Thus, we have a set of formulas, to describe any \textit{deformed pose} of the rigid body, as a rigid motion of its template.\\
 this is the reason why in the template ( rest pose ) ... you translate by $\bar{c}_cm$
\section{Impulse based treatement of $F_{ext}$}
Additionally, we will treat the displacement of the two rigid bodies, as an impulse $J = F_{ext} h$, where $h$ is a predefined timestep.\\ Let us add a (nonconstant) external force that acts during each timestep interval.\\

%%%%%%%%%%%%%%%%%%%%%
%%% THE ALGORITHM %%%
%%%%%%%%%%%%%%%%%%%%%
\newpage
\section{Flow of the Algorithm}
\begin{enumerate}
	\item Sum up $F_{ext}$, over each boundary edge of the rigid body. 
	\item Convert each $F_{ext}$ to $F_{config}$, via the formalu $F_{config} = dv^{T}F_{ext}$
	\item Sum up $F_{config}$ over tehs sytem
	\item Update $[c,\theta]$ using impulse formulas.
\end{enumerate}




%%%%%%%%%%%%%%%%%%%
%%% NEEDED CODE %%%
%%%%%%%%%%%%%%%%%%%
\section{Pieces of Needed Code}
\begin{enumerate}
	\item $dv = [I -R_{\theta}[\bar{v}]_x T(\theta)$
	\item $R_{\theta}$.
	\item Cross Product Matrix
	\item $T$ matrix % what is this $T$ matrix ... again!
\end{enumerate}







	







%%%%%%%%%%%%%%%%%
%%% Questions %%%
%%%%%%%%%%%%%%%%%
\newpage
\section{Questions}
	\begin{enumerate}
		\item What makes SGD better than Grad descent, or other learning techniques? ( reference Udacity DL lectures here? )
		\item What is the set of main drawbacks to focus here, on SGD? 
		\item Can I capture a notion of the runtime of my algorithms here?  
		\item Do I need to be incredibly specific about the classess of Rigid Bodies I am considering? 
		\item What are the characteristics of the trimeshes I am considering? I presume the following :: simple ( non-intersecting ), single, connected ( ... maybe ), elongated boundaries ( ... maybe ), not neccessarily convex, constant density $\rho$.  
		\item Assuming that our inputs are already presented as valid trimeshes, means that I can assume that the user already performed a triangulation, and possibly, Poisson Surface Reconstruction, from point-cloud data.
		\item For the COM/centroid formulation, can I refer to reader either to an appendix section, or outside readings? 
		\item How do I eliminate typesetting equations with a corresponding (index)?
		\item Do I need to also mention background, with the Rodrigues's formula too? UNSURE of this atm!
		\item Just to verify, the Lagrangian fails to make sense, in this physical setting, since we are not dealing with rigid bodies in free motion, that follow d L'ambert's principle, but rather, the motio of rigid bodies, constrained by the interpolating surface. THUS, we work with the notion of treating the external forces, over a small timestep, as impulses on the system. BUT WAIT ... technically, the formula for converting from $F_{ext}$ to $F_{config}$ mandates a potential energy, right? ... or perhaps not!
		\item What is the correct formulation of impulses, to use in this setting?
		\item How do I ensure that when I apply $F_{ext}$ or $J$ to the rigid bodies, it does not result in a penetrating rigid alignment? Sure, the stopping condition is based on the surface area energy metric ... but I don't think that's a guarantee, right?? unsure!
	\end{enumerate}

% a lot of the set up, for this approach , belongs to an appendix section ( part b? ... hmmm ... )
% just get the approach working for COM caes only, with the mirror planes set up
% you need some discretizeed $\omega$ in the system, sadly, since the $\theta$ update is based on updating the $\omega$.

\end{document}
